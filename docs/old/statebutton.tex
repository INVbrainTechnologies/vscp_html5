% This file was converted from HTML to LaTeX with
% gnuhtml2latex program
% (c) Tomasz Wegrzanowski <maniek@beer.com> 1999
% (c) Gunnar Wolf <gwolf@gwolf.org> 2005-2010
% Version : 0.4.
\documentclass{article}
\begin{document}

        
        \section*{vscpws\_stateButton}
        \par \textit{This information is preliminary and can change at any time.}
        
        \par 
        The \textit{vscpws\_stateButton} is a button that can be used as a state button or
        as a stateless button. It has the ability to send an event when it changes
        it state (one for each state) or for a stateless button one when it is pressed
        and one when released. Also the button can change it's state according to 
        incoming events or a monitored VSCP Daemon variable value. The button can 
        have one of several appearances shown below or a custom appearance defined 
        by the user.
        
        \par 
        To assign a monitored variable to the button is a good solution if a state 
        should be monitored over
        time. Instead of waiting for the next event that tell the button state a
        variable is read from the daemon which is kept persistent and up to date to
        get the state. This variable is written/changed in the decision matrix or
        by a user or an application through the tcp/ip interface of the daemon
        and can thus be influenced and controlled by some simple or advanced criteria.
        This also can be used to monitor the state for a specific condition by
        monitor a variable and just show the state on the page not allowing clicking
        the button. Typical scenarios can be to indicate an open window, an alarm 
        condition or things like that.
        
        \par 
        The decision matrix of the VSCP daemon is also perfect to use if one wants 
        to have a button that sends several events. This can be anything of course. 
        Some dimmer commands combined with some on/offs and some curtain roll downs. 
        This is then just added to the 
        decision matrix and triggered on the event sent from the button. Typical use
        is to control for example any lamps from one button to set a specified scenery
        for a house or a stage.
        
        \par 
        A \textbf{state button} as default send \textbf{CLASS1.CONTROL,TurnOn} 
        \textit{index=0, zone=0, subzone=0} when 
        activated and \textbf{CLASS1.CONTROL,TurnOff} \textit{index=0, zone=0, subzone=0} when 
        inactivated. The button will change it's state either when the monitored
        variable change its state or go to the active state when an 
        \textbf{CLASS1.INFORMATION, On-event} \textit{index=0, zone=0, subzone=0}is  
        received and go to it's inactive state when a 
        \textbf{CLASS1.INFORMATION,off-event} \textit{index=0, zone=0, subzone=0} is received.
        
        A \textbf{stateless button} as default send \textbf{CLASS1.CONTROL,TurnOn} 
        \textit{index=0, zone=0, subzone=0} when 
        pressed and \textbf{CLASS1.CONTROL,TurnOff} \textit{index=0, zone=0, subzone=0} when 
        released.
        \par 
        Creating a button is typically done with    
        \begin{verbatim}
            var btn1 = new vscpws_stateButton( "ws://192.168.1.20:7681", 
                                                "buttonKitchen");
        \end{verbatim}
        which will create a default state button of type=0 looking like this. 
        \par 
            
            \\
        
        \textit{"buttonKitchen"} referes
        to a canvas explained below.
        
        \subsection*{Parameters}
        \begin{verbatim}
        vscpws_stateButton( url,              // url to VSCP websocket i/f
                               canvasName,    // Where it should be placed
                               bLocal,        // Local visual control      
                               btnType,       // Button type   
                               bNoState,      // True for nonstate button 
                               bDisable,      // Disable user interactions
                               customupsrc,   // Custom up image
                               customdownsrc) // Custom down image
        \end{verbatim}                       
        \subsubsection*{url}
        This is the url to the websocket server. This typically is on the form
        \begin{verbatim}
            "ws://192.168.1.20:7681"
        \end{verbatim}
        With all widgets having there own url specified for the websocket server 
        it is possible to create web pages that control nodes/units/hardware that
        are located in different locations from the same page. 
        \subsubsection*{canvasName}
        This is the name of the canvas element where the button should be placed.
        Typically this is defined on the form
        \begin{verbatim}
        &LT;\textbf{canvas} id="buttonLivingroom1"
           style="z-index: 3;
           position:absolute;
           left:290px;
           top:430px;" height="30px" width="22px"&GT;
        &LT;/canvas&GT;   
        \end{verbatim}     
        The \textit{id} is the parameter that goes for the \textit{canvasName}. This
        name is also used to create the instance name for the button and the name
        set is preceded with \textit{\&Dot;vscpctrl\_\&Dot;}. The canvas specifies the
        position and the size for the button. Also z-order is possible to define 
        so that objects can be placed behind, ort partially behind each other
        to get nice visual effects.
        \subsubsection*{bLocal}
        If set to true visual indication is handled locally. For a state button
        appearance will be changed on every click and for a stateless button
        appearance will be changed when the button is pressed and then again when 
        it is released. It is better to have the visual appearance of the buttons
        controlled by external events as it also a confirmation that the sent event 
        actually did what it was intended to do. In some situations however it may be
        good to handle this locally for some reason. Default=false
        \subsubsection*{btnType}
        This is the visual appearances of the button. Check for possible values 
        below. \textbf{-1} is reserved for a user defined button. When used one has
        to supply a path to an image for both the "down" and the "up" state for the
        button in the \textit{customupsrc} and \textit{customdownsrc} parameters.
        parameters. Default = 0.
        \subsubsection*{bNoState}
        Set to true to have a stateless button. This button will send one event
        when pressed and one when released. The default is to send \textbf{CLASS1.CONTROL,TurnOn} 
        \textit{index=0, zone=0, subzone=0} when 
        pressed and \textbf{CLASS1.CONTROL,TurnOff} \textit{index=0, zone=0, subzone=0} when released.
        Default = false;
        \subsubsection*{bDisable}
        Set to true to disable user interactions. Button clicks will no longer
        work. This is useful if one instead wants to monitor a variable. Default is false. 
        \subsubsection*{customupsrc}
        This is the path to the "up" image for a custom button when type is -1.
        \subsubsection*{customdownsrc}
        This is the path to the "down" image for a custom button when type is -1.
        
        
        
        \subsection*{Methods}
        
        \subsection*{setOnTransmittEvent}
        Set the event that is sent when a state button is activated or a stateless
        button is pressed down.
        \begin{verbatim}
            setOnTransmittEvent(vscpclass,  // VSCP class
                                  vscptype, // VSCP type
                                  data,     // Array with databytes   
                                  guid)     // GUID to use
        \end{verbatim}
        \paragraph*{vscpclass}
        VSCP class to use for the event. Default=30 CLASS1.CONTROL
        \paragraph*{vscptype}
        VSCP type to use for the event. Default=5 as for CLASS1.CONTROL TurnOn-Event.
        \paragraph*{data}
        Array with data bytes to use for the event. Its the responsibility
        of the caller to use the correct number of data bytes. Default
        is "0,0,0" meaning index=0, zone=0, subzone=0,
        \paragraph*{guid}
        This is the sending GUID. Normally one just use the GUID of the sending 
        interface of the VSCP daemon. So the default is 00:00:00:00:00:00:00:00:00:00:00:00:00:00:00:00
        or "-" with is a short hand version for the long form. Default="-"
        
        If you don't want an event to be sent for the on-state set vscpclass=0.
        
        \subsection*{setOffTransmittEvent}
        Set the event that is sent when a state button is activated or a stateless
        button is pressed down.
        \begin{verbatim}
            setOffTransmittEvent(vscpclass,  // VSCP class
                                  vscptype,  // VSCP type
                                  data,      // Array with databytes   
                                  guid)      // GUID to use
        \end{verbatim}
        \paragraph*{vscpclass}
        VSCP class to use for the event. Default=30 CLASS1.CONTROL
        \paragraph*{vscptype}
        VSCP type to use for the event. Default=6 as for CLASS1.CONTROL TurnOff.
        \paragraph*{data}
        Array with data bytes to use for the event. Its the responsibility
        of the caller to use the correct number of data bytes. Default
        is "0,0,0" meaning index=0, zone=0, subzone=0,
        \paragraph*{guid}
        This is the sending GUID. Normally one just use the GUID of the sending 
        interface of the VSCP daemon. So the default is 00:00:00:00:00:00:00:00:00:00:00:00:00:00:00:00
        or "-" with is a short hand version for the long form. Default="-"
        
        If you don't want an event to be sent for the off-state set vscpclass=0.
        
        \subsection*{setOnReceiveEvent}
        Set the event that should be received to set the visual indication of
        a button to it's on state.
        \begin{verbatim}
            setOnReceiveEvent(vscpclass,    // VSCP class
                                  vscptype, // VSCP type
                                  data,     // Array with databytes   
                                  guid)     // GUID to use
        \end{verbatim}
        \paragraph*{vscpclass}
        VSCP class that the incoming event must have. Default=20 CLASS1.INFORMATION
        \paragraph*{vscptype}
        VSCP type that the incoming event must have. Default=3 as of the 
        CLASS1.INFORMATION On-Event.
        \paragraph*{data}
        Array with data bytes that the event should have. Set a data byte to -1
        for a don't care. Default
        is "-1,0,0" meaning index=-1, zone=0, subzone=0,
        \paragraph*{guid}
        This is the GUID the receiving event should have. Just leave undefined
        if the GUID should not be checked. Default=""
        
        \subsection*{setOffReceiveEvent}
        Set the event that should be received to set the visual indication of
        a button to it's on state.
        \begin{verbatim}
            setOffReceiveEvent(vscpclass,    // VSCP class
                                  vscptype,  // VSCP type
                                  data,      // Array with databytes   
                                  guid)      // GUID to use
        \end{verbatim}
        \paragraph*{vscpclass}
        VSCP class that the incoming event must have. Default=20 CLASS1.INFORMATION
        \paragraph*{vscptype}
        VSCP type that the incoming event must have. Default=3 as of the 
        CLASS1.INFORMATION Off-Event.
        \paragraph*{data}
        Array with data bytes that the event should have. Set a data byte to -1
        for a don't care. Default
        is "-1,0,0" meaning index=-1, zone=0, subzone=0,
        \paragraph*{guid}
        This is the GUID the receiving event should have. Just leave undefined
        if the GUID should not be checked. Default=""
        
        
        \subsection*{setOnTransmittZone}
        As it is common to use one of the types in CLASS1.CONTROL as on-event
        for a button this is a convenient method to use to change the default
        zone/subzone for the outgoing event.
        \begin{verbatim}
            setOnTransmittZone(zone,     // zone
                                subzone, // subzone
        \end{verbatim}
        \paragraph*{zone}
        Zone to direct event to. 255 is all.
        \paragraph*{subzone}
        Subzone to direct event to. 255 is all.
        
        \subsection*{setOffTransmittZone}
        As it is common to use one of the types in CLASS1.CONTROL as off-event
        for a button this is a convenient method to use to change the default
        zone/subzone for the outgoing event.
        \begin{verbatim}
            setOffTransmittZone(zone,    // zone
                                subzone, // subzone
        \end{verbatim}
        \paragraph*{zone}
        Zone to direct event to. 255 is all.
        \paragraph*{subzone}
        Subzone to direct event to. 255 is all.
        
        
        
        \subsection*{setOnReceiveZone}
        As it is common to use one of the types in CLASS1.INFORMATION as a trigger
        for the on-state of 
        a button this is a convenient method to use to change the default
        zone/subzone for the incoming event.
        \begin{verbatim}
            setOnReceiveZone(zone,     // zone
                              subzone, // subzone
        \end{verbatim}
        \paragraph*{zone}
        Zone to direct event to. 255 is all.
        \paragraph*{subzone}
        Subzone to direct event to. 255 is all.
        
        
        \subsection*{setOffReceiveZone}
        As it is common to use one of the types in CLASS1.INFORMATION as a trigger
        for the off-state
        of a button this is a convenient method to use to change the default
        zone/subzone for the incoming event.
        \begin{verbatim}
            setOffReceiveZone(zone,     // zone
                              subzone,  // subzone
        \end{verbatim}
        \paragraph*{zone}
        Zone to direct event to. 255 is all.
        \paragraph*{subzone}
        Subzone to direct event to. 255 is all.
        
        \subsection*{setMonitorVariable}
        With this method one can set a VSCP daemon boolean variable that should
        be monitored with a specific interval. If the variable is true the state for
        the state button will be set to true and vice versa. The variable is also
        updated the other way around
        If the button is down the variable will 
        be set to true and vice versa. If the variable does not exist it will be 
        created.
        \begin{verbatim}
            setMonitorVariable(variablename, // variable name
                                interval,    // monitoring interval in milliseconds
        \end{verbatim}
        \paragraph*{variablename}
        The name for the VSCP daemon boolean variable.
        \paragraph*{interval}
        The interval in milliseconds between variable reads. Set to zero to disable.
        Default=1000 (one second).
        To test try to set the variable name to "statbutton1" and issue
        \begin{verbatim}
            variable write statebutton1,,,true
        \end{verbatim}
        and
        \begin{verbatim}
            variable write statebutton1,,,false
        \end{verbatim}
        in the VSCP daemon TCP/IP interface to see the change the state for the state 
        button in your browser page.
        
                
        \subsection*{draw}
        Draw the widget. Normally no need to use.
        \begin{verbatim}
            draw() // Draw the widget
        \end{verbatim}
        
        \subsection*{setValue}
        Set the value for the widget. The state will also be 
        \begin{verbatim}
            setValue(value,       // Boolean value for state
                        bUpdate ) // True (default) if a draw should be done
                                      after value is set.
        \end{verbatim}
        \paragraph*{value}
        Boolean value that set the state of the widget.
        \paragraph*{bUpdate}
        True (default) if a draw should be done after value is set.
        
        
        \subsection*{Appearances types}
        The images below is shown in there default size. Most of the buttons 
        is designed for a different size and with an appropriate background colour. 
        
            \subsubsection*{Type = 0 Default}    
            
            \\
        
        
        
            \subsubsection*{Type = 1}    
            
            \\
        
        
        
            \subsubsection*{Type = 2}    
            
            \\
        
        
        
            \subsubsection*{Type = 3}    
            
            \\
        
        
        
            \subsubsection*{Type = 4}    
            
            \\
        
        
        
            \subsubsection*{Type = 5}    
            
            \\
        
        
        
            \subsubsection*{Type = 6}    
            
            \\
        
        
        
            \subsubsection*{Type = 7}    
            
            \\
        
        
        
            \subsubsection*{Type = 8}    
            
            \\
        
        
        
            \subsubsection*{Type = 9}    
            
            \\
        
        
        
            \subsubsection*{Type = 10}    
            
            \\
        
        
        
            \subsubsection*{Type = 11}    
            
            \\
        
        
        
            \subsubsection*{Type = 12}   
            
            
            \\
        
        
        
            \subsubsection*{Type = 13}    
            
            
            \\
        
        
        
            \subsubsection*{Type = 14}    
            
            
            \\
        
        
        
            \subsubsection*{Type = 15}    
            
            \\
        
        
        
            \subsubsection*{Type = 16}    
            
            \\
        
        
        
            \subsubsection*{Type = 17}    
            
            \\
        
        
        
            \subsubsection*{Type = 18}    
            
            \\
        
        
        
            \subsubsection*{Type = 19}    
            
            \\
        
        
        
            \subsubsection*{Type = 20}    
            
            \\
        
        
        
            \subsubsection*{Type = 21}    
            
            \\
        
        
        
            \subsubsection*{Type = 22}    
            
            \\
        
        
        
            \subsubsection*{Type = 23}    
            
            \\
        
        
        
            \subsubsection*{Type = 24}    
            
            \\
        
        
        
            \subsubsection*{Type = 25}    
            
            \\
        
        
        
            \subsubsection*{Type = 26}    
            
            \\
        
        
        
            \subsubsection*{Type = 27}    
            
            \\
        
        
        
            \subsubsection*{Type = 28}    
            
            \\
        
        
        
            \subsubsection*{Type = 29}    
            
            \\
        
        
        
            \subsubsection*{Type = 30}    
            
            \\
        
        
        
            \subsubsection*{Type = 31}    
            
            \\
        
        
        
            \subsubsection*{Type = 32}    
            
            \\
        
        
        
            \subsubsection*{Type = 33}    
            
            \\
        
        
        
            \subsubsection*{Type = 34}    
            
            \\
        
        
        
            \subsubsection*{Type = 35}    
            
            \\
        
        
        
            \subsubsection*{Type = 36}    
            
            \\
        
        
        
            \subsubsection*{Type = 37}    
            
            \\
        
        
        
            \subsubsection*{Type = 38}    
            
            \\
        
        
        
            \subsubsection*{Type = 39}    
            
            \\
        
        
        
            \subsubsection*{Type = 40}    
            
            \\
        
        
        
            \subsubsection*{Type = 41}    
            
            \\
        
        
        
            \subsubsection*{Type = 42}    
            
            \\
        
        
        
            \subsubsection*{Type = 43}    
            
            \\
        
        
        
            \subsubsection*{Type = 44}    
            
            \\
        
        
        
            \subsubsection*{Type = 45}    
            
            \\
        
        
        
            \subsubsection*{Type = 46}    
            
            \\
        
        
        
            \subsubsection*{Type = 47}    
            
            \\
        
        
        
            \subsubsection*{Type = 48}    
            
            \\
        
        
        
            \subsubsection*{Type = 49}    
            
            \\
        
        
        
            \subsubsection*{Type = 50}    
            
            \\
        
        
        
            \subsubsection*{Type = 51}    
            
            \\
        
        
        
            \subsubsection*{Type = 52}    
            
            \\
        
        
        
            \subsubsection*{Type = 53}    
            
            \\
        
        
        
            \subsubsection*{Type = 54}    
            
            \\
        
        
        
            \subsubsection*{Type = 55}    
            
            \\
        
        
        
            \subsubsection*{Type = 56}    
            
            \\
        
        
        
            \subsubsection*{Type = 57}    
            
            \\
        
        
        
            \subsubsection*{Type = 58}    
            
            \\
        
        
        
            \subsubsection*{Type = 59}    
            
            \\
        
        
        
            \subsubsection*{Type = 60}    
            
            \\
        
        
        
            \subsubsection*{Type = 61}    
            
            \\
        
        
        
            \subsubsection*{Type = 62}    
            
            \\
        
        
        
            \subsubsection*{Type = 63}    
            
            \\
        
        
        
            \subsubsection*{Type = 64}    
            
            \\
        
        
        
            \subsubsection*{Type = 65}    
            
            \\
        
        
        
            \subsubsection*{Type = 66}    
            
            \\
        
        
        
            \subsubsection*{Type = 67}    
            
            \\
        
        
        
            \subsubsection*{Type = 68}    
            
            \\
        
        
        
            \subsubsection*{Type = 69}    
            
            \\
        
        
        
            \subsubsection*{Type = 70}    
            
            \\
        
        
        
            \subsubsection*{Type = 71}    
            
            \\
        
        >
        
            \subsubsection*{Type = 72}    
            
            \\
        
        
        
            \subsubsection*{Type = 73}    
            
            \\
        
        
        
            \subsubsection*{Type = 74}    
            
            \\
        
        
        
            \subsubsection*{Type = 75}    
            
            \\
        
        
        
            \subsubsection*{Type = 76}    
            
            \\
        
        
        
            \subsubsection*{Type = 77}    
            
            \\
        
        
        
            \subsubsection*{Type = 78}    
            
            \\
        
        
        
            \subsubsection*{Type = 79}    
            
            \\
        
        
        
            \subsubsection*{Type = 80}    
            
            \\
        
        
        
            \subsubsection*{Type = 81}    
            
            \\
        
        
        
            \subsubsection*{Type = 82}    
            
            \\
        
        
        
            \subsubsection*{Type = 83}    
            
            \\
        
        
        
            \subsubsection*{Type = 84}    
            
            \\
        
        
        
            \subsubsection*{Type = 85}    
            
            \\
        
        
        
            \subsubsection*{Type = 86}    
            
            \\
        
        
        
            \subsubsection*{Type = 87}    
            
            \\
        
        
        
            \subsubsection*{Type = 88}    
            
            \\
        
        
        
            \subsubsection*{Type = 89}    
            
            \\
        
    \end{document}
